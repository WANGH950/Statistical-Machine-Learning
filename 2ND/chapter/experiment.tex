\section{开始实验}
首先加载需要用到的包
\begin{lstlisting}
import torch
import torch.nn as nn
from collections import OrderedDict
import copy
import time
from sklearn import datasets
import matplotlib.pyplot as plt
torch.manual_seed(100)
\end{lstlisting}
我们的所有实验都基于PyTorch神经网络框架实现, 该框架不仅包含了数值计算包numpy中的几乎所有工具, 还提供了编写神经网络的基本工具, 起到了类似于建房子时提供"砖头"和"水泥"的作用.
本次实验中所有需要计算梯度的地方都通过PyTorch中提供的自动微分算法实现, 需要使用的优化算法也由PyTorch框架提供.
我们将随机数种子设置为100, 以保证实验结果可以复现.

我们使用sklearn中的datasets自动下载并加载数据, 之后将其转化为torch.tensor类型的数据, 以便后续使用torch的工具接口进行处理.
\begin{lstlisting}
# 加载数据
data =  datasets.load_wine()
data_features = torch.tensor(data.data)
data_labels = torch.tensor(data.target)[:,None]
data_set = torch.cat([data_features,data_labels],dim=1)
\end{lstlisting}

我们使用torch中提供的划分数据集的工具, 将数据按照6:4随机划分为训练集和测试集.
\begin{lstlisting}
length = data_set.shape[0]
alpha = 0.6
train_len = int(length*alpha)
test_len = length - train_len
train_data, test_data = torch.utils.data.random_split(data_set,[train_len,test_len])
train_data = torch.tensor([item.numpy() for item in train_data])
test_data = torch.tensor([item.numpy() for item in test_data])
\end{lstlisting}
