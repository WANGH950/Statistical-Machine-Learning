\documentclass{lzureport}

\usepackage{amsmath}
\usepackage{caption}
\usepackage[dvipsnames]{xcolor}  % 更全的色系
\usepackage{listings}  % 排代码用的宏包
\usepackage{hyperref}
\usepackage{graphicx}  %插入图片的宏包
\usepackage{float}  %设置图片浮动位置的宏包
\usepackage{subfigure}  %插入多图时用子图显示的宏包
\hypersetup{hidelinks,
	colorlinks=true,
	allcolors=black,
	pdfstartview=Fit,
	breaklinks=true}

%%%%%%%%%%%%%%%%%%%%%%%%%%%%%%%%%%%%%%%%
%% listings设置
%%%%%%%%%%%%%%%%%%%%%%%%%%%%%%%%%%%%%%%%
\lstset{
	language = Python,
	backgroundcolor = \color{yellow!10},    % 背景色:淡黄
	basicstyle = \small\ttfamily,           % 基本样式 + 小号字体
	rulesepcolor= \color{gray},             % 代码块边框颜色
	breaklines = true,                  % 代码过长则换行
	numbers = left,                     % 行号在左侧显示
	numberstyle = \small,               % 行号字体
	keywordstyle = \color{blue},            % 关键字颜色
	commentstyle =\color{green!100},        % 注释颜色
	stringstyle = \color{red!100},          % 字符串颜色
	frame = shadowbox,                  % 用(带影子效果)方框框住代码块
	showspaces = false,                 % 不显示空格
	columns = fixed,                    % 字间距固定
	%escapeinside={<@}{@>}              % 特殊自定分隔符:<@可以自己加颜色@>
	morekeywords = {as},                % 自加新的关键字(必须前后都是空格)
	deletendkeywords = {compile}        % 删除内定关键字;删除错误标记的关键字用deletekeywords删!
}


\major{计算数学}
\name{王恒}
\title{统计学习方法代码实现}
\stuid{220220934161}
\college{数学与统计学院}
\date{\zhtoday}
\expname{多种模型的葡萄酒品种分类任务}
\course{统计机器学习实验报告}


\begin{document}

\makecover

\begin{abstract}
    本文中, 我们实现了感知机的原型算法和对偶算法, 并完成了三点数据分类的题目. 同时, 我们使用对偶算法对MNIST数据集进行分类. 其中, 我们首先考虑将以$\{y:y=0,1,\cdots,9\}$为标签的多分类问题转化为以
    \begin{equation*}
        \hat{y} = \left\{\begin{aligned}
            &1,\qquad &y < 5,\\
            &-1,&y\geq 5.
        \end{aligned}\right.
    \end{equation*}
    为标签的二分类问题, 之后应用对偶算法分类.
    其次, 我们实现了k-近邻的kd-树的构造算法及搜索算法, 并在MNIST数据集上进行实验. 最后我们对实验结果进行了分析, 实验结果证明我们复现的算法是正确的, 并达到了预期效果.
    本实验报告的所有内容开源: \url{https://github.com/WANGH950/Statistical-Machine-Learning/tree/main/1ST }.
\end{abstract}

\textbf{关键词:} 感知机, k-近邻, kd-树, MNIST
\newpage

\thispagestyle{empty}
\tableofcontents
\newpage 
\setcounter{page}{1}

\section{实验代码}

\subsection{感知机学习算法}

\begin{lstlisting}[caption = 感知机原型算法实现]
	class Perception():
    def __init__(self, dim) -> None:
        # 构造函数
        # dim:   特征维度
        # w:     权重
        # b:     偏置项

        self.dim = dim
        self.w = np.zeros([dim])
        self.b = 0
    
    def train(self, data_set, epoch, learning_rate):
        # 训练模型

        for i in range(epoch):
            for (x,y) in data_set:
                if self.predict(x)*y <= 0:
                    self.w = self.w + learning_rate*x*y
                    self.b = self.b + learning_rate*y
            # 计算准确率
            acc = self.accuracy(data_set)
            print('epoch: ',i+1, 'accuracy: ', acc)
            # 早停条件
            if acc == 1:
                break
        print('Trining complete.')

    def predict(self, x):
        # 预测

        return np.sign(np.dot(self.w,x) + self.b)
    
    def accuracy(self, data_set):
        # 计算精度
        
        acc = 0
        for (x,y) in data_set:
            if self.predict(x)*y > 0:
                acc += 1
        return acc/len(data_set)
\end{lstlisting}

\begin{lstlisting}[caption = 感知机对偶算法实现]
	class PerceptionDual():
    def __init__(self, data) -> None:
        # 构造函数
        # data:     训练数据
        
        self.data = data
        self.N = len(data)
        self.x = np.array([xx for (xx,_) in data])
        self.y = np.array([yy for (_,yy) in data])
        self.alpha = np.zeros([self.N])
        self.b = 0
    
    def train(self, epoch, learning_rate):
        # 训练模型
        
        for i in range(epoch):
            for j in range(len(self.data)):
                if self.predict(self.data[j][0])*self.data[j][1] <= 0:
                    self.alpha[j] = self.alpha[j] + learning_rate
                    self.b = self.b + learning_rate*self.data[j][1]
            # 计算准确率
            acc = self.accuracy(self.data)
            print('epoch: ',i+1, 'accuracy: ', acc)
            # 早停条件
            if acc == 1:
                break
        print('Trining complete.')

    def predict(self, x):
        # 预测
        return np.sign(np.dot(np.dot(self.x,x),self.alpha*self.y) + self.b)
    
    def accuracy(self, data_set):
        # 计算精度

        acc = 0
        for (x,y) in data_set:
            if self.predict(x)*y > 0:
                acc += 1
        return acc/len(data_set)
\end{lstlisting}

\subsection{k-近邻算法}

\begin{lstlisting}[caption = kd-树构造算法和搜索算法实现]
	class Node():
    def __init__(self, value, data, label) -> None:
        # 构造函数
        # value:        节点的划分超平面参数
        # data:         落在超平面上的数据点
        # label:        落在超平面上的数据点对应的标签

        self.value = value
        self.data = data
        self.label = label
        self.left = None
        self.right = None
    
    def set_left(self, node):
        # 设置左子节点

        if node != None:
            self.left = node

    def set_right(self, node):
        # 设置右子节点

        if node != None:
            self.right = node

class KDTree():
    def __init__(self) -> None:
        # 构造函数
        # 用于存储KDTree,支持直接实例化对象时直接输入一个kd-树
        self.root = None
    
    def create(self, data, label, j = 0):
        # 递归构造平衡KD树

        num, k = data.shape
        if num == 0:
            return None
        else:
            l = j % k
            ind_sorted = np.argsort(data[:,l])
            ind_median = ind_sorted[num//2]
            value_ = int(np.median(data[ind_median,l]))
            data_ = data[data[:,l]==value_]
            label_ = label[data[:,l]==value_]
            node = Node(
                value=value_,
                data=data_,
                label=label_
            )
            node.set_left(
                self.create(
                    data=data[data[:,l]<value_],
                    label=label[data[:,l]<value_],
                    j=j+1
                )
            )
            node.set_right(
                self.create(
                    data=data[data[:,l]>value_],
                    label=label[data[:,l]>value_],
                    j=j+1
                )
            )
            if j == 0:
                self.root = node
            else:
                return node
    
    def search(self, x, j = 0, node = None):
        # 递归搜索KD树

        if self.root == None:
            print("You haven't created a KDTree yet.")
            return None
        if j == 0:
            node = self.root
        k = x.shape[0]
        l = j % k
        # 叶子节点停止条件
        if self.is_leaf(node):
            distance = np.linalg.norm(x-node.data,2,1)
            index = np.argmin(distance)
            return node.data[index], node.label[index]
        else:
            # 计算当前节点中的最近数据点
            distance = np.linalg.norm(x-node.data,2,1)
            min_distance = np.min(distance)
            index = np.argmin(distance)
            nearest = node.data[index]
            label = node.label[index]
            # 递归计算子节点的最近数据点,并比较
            if x[l] < node.value and node.left != None:
                nearest_, label_ = self.search(
                    x = x,
                    j = j+1,
                    node = node.left
                )
                if np.linalg.norm(x-nearest_,2) < min_distance:
                    nearest = nearest_
                    label = label_
            elif x[l] > node.value and node.right != None:
                nearest_, label_ = self.search(
                    x = x,
                    j = j+1,
                    node = node.right
                )
                if np.linalg.norm(x-nearest_,2) < min_distance:
                    nearest = nearest_
                    label = label_
            return nearest, label

    def is_leaf(self, node: Node):
        # 判断是否是叶子节点
        
        if node.left != None or node.right != None:
            return False
        else:
            return True
\end{lstlisting}

\section{实验结果分析}
\zhlipsum[3]
\subsection{感知机学习算法结果分析}

\subsection{k-近邻算法结果分析}

\newpage
\thispagestyle{empty}
\bibliography{books}
\newpage

\appendix

\section{串行部分全部实验结果}

图或者代码放上来。

\section{并行部分全部实验结果}

图或者代码放上来。

\end{document}